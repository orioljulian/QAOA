\documentclass{report}
\usepackage[spanish]{babel}
\usepackage{graphicx}
\usepackage{subfigure}
\usepackage{amsmath}
\usepackage{hyperref}  % hyperlinks
\graphicspath{{/home/vian/0_uam/1_TFG/latex/img/}}

\title{Estudio de la aplicación de algoritmos cuánticos de optimización con las tecnologías cuánticas actuales}

\begin{document}

\maketitle{}
\tableofcontents{}

\newpage

\chapter{Introducción}
Explicación sobre la computación cuántica, concretamente orientada a problemas de optimización. Mencionar su posible supremacía para la ejecución de ciertos problemas con respecto a la computación clásica. También explicar el concepto de era NISQ y la utilidad durante la misma de algoritmos híbridos.

Explicar diferencia entre Quantum Annealing y QAOA (D-Wave vs ordenadores cuánticos generalistas para resolver problemas de optimización)

\chapter{Estado del arte}
El artículo del que se ha partido.

Mostrar también aquí el resultado de la gráfica que ahora mismo está en la sección \ref{sec:integracion_pruebas_y_resultados}.

\chapter{Diseño}
Formular el problema de \textbf{shortest path}
Cómo se aplica el algoritmo QAOA a un problema de \textbf{shortest path} genérico. O problema de optimización genérico? Igual esto mejor.

Cómo pasar de problema de tipo Ising a tipo QUBO, luego explicar esto cómo se aplica en QAOA y D-Wave

También hablar sobre la formulación de MAX-CUT, para el caso de Qiskit

\chapter{Desarrollo}
Aplicar el algoritmo QAOA a cada uno de los tres grafos. Construir las funciones de coste y mostrar el hamiltoniano resultante.

Cómo llamar a los grafos?? Estaría bien decir primero, segundo y tercero o algo así?
\section{Grafo de planificación de red simple}

\section{Grafo de MAX-CUT simple}

\section{Grafo de doble operador Pauli}  % TODO: Sacado del artículo, ver el porqué

\chapter{Integración, pruebas y resultados}
\label{sec:integracion_pruebas_y_resultados}

\section{Primer grafo}

\subsection{QAOA}

\subsubsection{Simulador}

\subsubsection{Ordenador cuántico real}

\subsection{Quantum Annealing}

\section{Tutorial de Qiskit}
Aquí se debería mostrar el problema aplicado a D-Wave también?

\section{Zhiqiang grafo}

\subsection{QAOA}

\subsubsection{Simulador}

\subsubsection{Ordenador cuántico real}

\subsection{Quantum Annealing}

\chapter{Conclusiones y trabajo futuro}

\chapter{Bibliografía}

\chapter{Apéndice}

\section{Cómo ejecutar con Runtime}
\section{Por qué funciona el método de pasar de problema de tipo Ising a tipo QUBO}


\end{document}