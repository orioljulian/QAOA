En definitiva, el circuito de QAOA está basado en un estado inicial $\ket{\psi_0}$ y un circuito parametrizado (también llamado circuito ansatz)
$U(B, \beta_{p-1})U(C, \gamma_{p-1}) \ldots U(B, \beta_0)U(C, \gamma_0)$, cuyo estado resultante se denomina $\lvert \psi(\vec{\beta}, \vec{\gamma}) \rangle$.

Como ya ha sido mencionado en la introducción, QAOA es un algoritmo híbrido, por lo que tiene una parte ejecutada en un computador cuántico y otra en un computador clásico. La medición de $\lvert \psi(\vec{\beta}, \vec{\gamma}) \rangle$ corresponde a la primera, mientras que la variación de $\vec{\beta}$ y $\vec{\gamma}$ es realizada por un optimizador clásico.

Así, la estructura general de QAOA es la siguiente~\cite{qaoa_paper_original}:

\begin{enumerate}
\item Inicializar $\vec{\beta}$ y $\vec{\gamma}$ a valores arbitrarios según el número de capas $p$.
\item\label{it:3-algoritmo_repetir}
  Repetir hasta que se cumpla cierto criterio de convergencia, obteniendo $\vec{\beta_{opt}}, \vec{\gamma_{opt}}$
  \begin{enumerate}
  \item\label{it:3-algoritmo_construir_circuito} Construir el circuito correspondiente al estado $\lvert \psi(\vec{\beta}, \vec{\gamma})\rangle$

  \item\label{it:3-algoritmo_ejecutar_circuito}
    Evaluar el estado en la base computacional.
    Esto, por el funcionamiento del modelo orientado a circuitos, devuelve una estimación de la probabilidad de cada uno de los estados de la base $\ket{x}$.

  \item\label{it:3-algoritmo_valor_esperado}
    Computar el valor esperado de $C$: $\langle \psi(\vec{\beta}, \vec{\gamma}) \rvert C \lvert \psi(\vec{\beta}, \vec{\gamma}) \rangle$.
    De forma equivalente, se puede obtener este valor de manera clásica calculando la media ponderada del coste en $f(x)$ de los estados obtenidos según su probabilidad.

    \begin{align}
      \sum_{\ket{x}} f(x)*P(\ket{x})
    \end{align}

    Donde $f(x)$ es el coste de la función para un valor $x$ obtenido en la ejecución del paso anterior y $P(\ket{x})$ es la probabilidad estimada de obtener x en dicha ejecución.

    El valor esperado obtenido representa una métrica de cómo de bueno es el resultado de la ejecución del circuito.
    A menor sea este valor, más cercano está $\lvert \psi(\vec{\beta}, \vec{\gamma})\rangle$ al estado fundamental.

  \item Hallar nuevos $\vec{\beta_{new}}, \vec{\gamma_{new}}$ usando un algoritmo clásico de optimización y volver al \textit{paso~\ref{it:3-algoritmo_repetir}}.
  \end{enumerate}

\item Evaluar $\lvert \psi(\vec{\beta_{opt}}, \vec{\gamma_{opt}}) \rangle$ para obtener el estado $x$ con mayor probabilidad.
  Este valor aproximará $\min_{x}f(x)$.

\end{enumerate}

Para el estado construido se cumple que para un número de capas $p \rightarrow \infty$:

\begin{align}
  \min_{\vec{\beta}, \vec{\gamma}} \langle \psi(\vec{\beta}, \vec{\gamma}) \rvert C \lvert \psi(\vec{\beta}, \vec{\gamma}) \rangle = \min_{x} f(x)
\end{align}

Esto es equivalente a decir que el estado fundamental (sección~\ref{sec:8-concepto-estado_fundamental}) del sistema corresponde al valor mínimo de la función de coste.

De esta forma, el optimizador clásico llamará en repetidas iteraciones a una función con una entrada $(\vec{\beta}, \vec{\gamma})$ que devolverá como salida un valor mayor o igual al mínimo de la función de coste.
Este resultado es el que el optimizador debe minimizar.

En este caso el optimizador clásico que se utilizará es COBYLA (\textit{Constrained Optimization BY Linear Approximation}), de la librería de \textit{Python} \textit{scipy.minimize}.


%%% Local Variables:
%%% mode: latex
%%% TeX-master: "../tfgtfmthesisuam"
%%% End:
