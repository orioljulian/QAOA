\section{Pasar de problemas de optimización combinatoria a QUBO\label{sec:3-problemas de optimizacion combinatoria}}

En la \textit{sección~\ref{sec:2-problemas de optimizacion combinatoria}} se explica que los problemas de optimización con restricciones deben ser transformados para su posterior procesamiento por los algoritmos tratados.
Una restricción puede ser definida como una igualdad de la forma $A(x) = B(x)$ que debe cumplir toda entrada $x$ de la función de coste para considerarse válida.
La forma de convertir un problema con restricciones a uno sin ellas es modificar la función de coste para que las incluya en su definición, por lo que en lugar de tener $f(x)$ como función de coste se tendrá:
$f'(x) = f(x) + P*{(A(x)-B(x))}^2$, donde

\begin{align*}
  P*{(A(x) - B(x))}^2 \begin{cases}
    = 0 \textnormal{ si se cumple la restricción } \\
    \ge P \textnormal{ en otro caso }
  \end{cases}
\end{align*}

El parámetro P se denomina \textit{modificador de Lagrange} y deberá tener un valor lo suficientemente grande como para que el castigo en caso de romper una restricción aumente lo suficiente el valor de la función de coste como para que sea mayor que cualquier otro resultado en el que no se rompa. Esto sería: $P > \max_x{f(x)}$.

\section{Circuito de QAOA\label{sec:3-circuito de qaoa}}
El sistema cuántico en el algoritmo se desarrolla sobre un espacio de Hilbert de $2^n$ dimensiones, donde \textit{n} es el número de bits de entrada en la función de coste clásica. Esto quiere decir que se tendrán tantos qubits como bits tenga la entrada de la función de coste $f(x)$.

La base computacional se representa como $\{\ket{x} : x \in {\{0, 1\}}^n\}$.

La idea general de QAOA se basa en preparar un estado \(\lvert \psi(\vec{\beta}, \vec{\gamma})\rangle\) tal que, con los valores adecuados \( (\vec{\beta_{opt}}, \vec{\gamma_{opt}}) \), el estado \(\lvert\psi(\vec{\beta_{opt}}, \vec{\gamma_{opt}})\rangle\) encuentre la solución al problema. Los parámetros de dicho estado son:
\begin{align*}
  \vec{\beta} = [\beta_0, \beta_1, \ldots , \beta_{p-1}] \\
  \vec{\gamma} = [\gamma_0, \gamma_1, \ldots , \gamma_{p-1}] \\
\end{align*}
Donde \textit{p} es el número de capas del circuito y $\beta_i, \gamma_i \in{[0, 2\pi]}$.

Este estado consta de tres componentes: el \textbf{estado inicial} ($\ket{\psi_0}$); y dos operadores denominados \textbf{problem hamiltonian} ($U(C, \gamma)$) y \textbf{mixing hamiltonian} ($U(B, \beta)$).
Estos se combinan de la siguiente forma:

\[
  \lvert\psi(\vec{\beta}, \vec{\gamma})\rangle = U(B, \beta_{p-1})U(C, \gamma_{p-1})U(B, \beta_{p-2})U(C, \gamma_{p-2}) \ldots U(B, \beta_0)U(C, \gamma_0) \ket{\psi_0}
\]

\subsection{Estado inicial}
El estado inicial del qubit se define como

\begin{align*}
  \ket{\psi_0} = \frac{1}{\sqrt{2^n}} \sum_x\ket{x}
  = {(\frac{1}{\sqrt{2}} (\ket{0} + \ket{1}))}^{\otimes n}
  = H^{\otimes n} \ket{0}^{\otimes n}
\end{align*}

Este estado inicial se construye añadiendo operadores de Hadamard a \textit{n} qubits inicializados a $\ket{0}$, lo que genera un estado equiprobable, es decir, donde la probabilidad de obtener una cadena dada de \textit{n} bits al medir el estado sería en cualquier caso $\frac{1}{2^n}$.

\subsection{Problem hamiltonian y mixing hamiltonian\label{sec:3-problem_y_mixing_hamiltonian}}

Los operadores se definen como:

\begin{align*}
  U(B, \beta) = e^{-i \beta B} \\
  U(C, \gamma) = e^{-i \gamma C} \\
\end{align*}

La operación \(U(X) = e^{-i X}\) sirve para garantizar la unitariedad del operador ya que, como se verá en su definición, \textit{B} y \textit{C} no son necesariamente unitarios
\footnote{En computación cuántica los operadores lineales deben ser unitarios. Tiene relación con la idea
  $\sum_{\ket{x}} {P(\ket{x})} = 1$,
  donde $\ket{x}$ itera sobre los vectores de la base computacional del sistema y
  \(P(\ket{x})\) se refiere a la probabilidad de medir i sobre dicha base.}.

Tanto la noción de hamiltoniano como la construcción de un operador unitario proceden de la ecuación de Schrödinger de la mecánica cuántica.
% TODO: Hablar en el apéndice sobre el segundo postulado del libro y su resolución. Añadir el QC-textbook como bibliografía.

\subsubsection{Operador \textit{B}}
\textit{B} se construye a partir de puertas Pauli-X ($\sigma^x$) y se define de la siguiente forma:

\begin{align*}
  &B = \sum_{i=0}^{n-1}\sigma^x_{i} \\
  &U(B, \beta) = e^{-i \beta B} = \prod_{j=0}^{n-1}e^{-i \beta \sigma^x_j} = \prod_{j=0}^{n-1}Rx_i(2*\beta)  % TODO: Citar aquí y definir Rx en el apéndice o a pie de página
\end{align*}

Como se puede ver únicamente depende de la cantidad de qubits del sistema.

\subsubsection{Operador \textit{C}\label{sec:3-operador c}}
\textit{C} debe ser un operador diagonal  % TODO: Citar del apéndice para explicar qué es un operador diagonal en |x>
en la base computacional \(\ket{x}\) con los valores de la función objetivo. De esta forma, siendo \(\ket{x}\) un vector en la base computacional se cumple que \(\forall \ket{x} = [x_1x_2 \ldots x_n], \bra{x}C\ket{x} = f(x)\). Se define a partir de la función de coste clásica del problema a resolver y se construye utilizando puertas Pauli-Z ($\sigma^z$).

Para obtener este operador se parte de una función de coste de la forma \(f(x)\), donde \(x=x_1x_2 \ldots x_n\) y \(x_i\in{0, 1}\).
Para construirlo la función utilizada debe estar en formato Ising, en lugar de la versión QUBO (caso de $f(x)$), lo cual se traduce en tener como entrada variables con valor \(\{-1, 1\}\), en lugar de \(\{0, 1\}\).
Para ello se debe realizar un cambio de variable:

\begin{align*}
  &f(x) \rightarrow C(z) \\
  &x_i \rightarrow \frac{1 - z_i}{2}
\end{align*}

De esta forma $z_i = 1$ si $x_i = 0$ y $z_i = -1$ si $x_i = 1$.
La forma de pasar de la función $C(z)$ al operador $C$ es sustituyendo las variables \(z_i\) por puertas \(\sigma^z_i\). La justificación de esto se encuentra en la \textit{sección~\ref{CAP:F_CLASICA_A_HAMILTONIANO}} del apéndice.

\section{Algoritmo de QAOA}
La estructura general del algoritmo es la siguiente:
\begin{enumerate}
\item Inicializar $\vec{\beta}$ y $\vec{\gamma}$ a valores arbitrarios.
\item\label{it:3-algoritmo_repetir}
  Repetir hasta que se cumpla cierto criterio de convergencia, obteniendo $\vec{\beta_{opt}}, \vec{\gamma_{opt}}$
  \begin{enumerate}
  \item Construir el circuito correspondiente al estado $\lvert\psi(\vec{\beta}, \vec{\gamma})\rangle$
  \item Medir el estado en la base computacional, lo que devuelve una cadena de bits ${\{0, 1\}}^n$
  \item Computar el valor esperado de $C$:\@$\langle \psi(\vec{\beta}, \vec{\gamma}) \rvert C \lvert \psi(\vec{\beta}, \vec{\gamma}) \rangle$.
    Esto puede ser calculado de manera equivalente evaluando la cadena de bits, obtenida de medir el estado, con la función de coste clásica $f(x)$.
  \item Hallar nuevos \(\vec{\beta_{new}}, \vec{\gamma_{new}}\) usando un algoritmo clásico de optimización y volver al \textit{paso~\ref{it:3-algoritmo_repetir}}.
  \end{enumerate}
\item Calcular el valor esperado: \( \langle \psi(\vec{\beta_{opt}}, \vec{\gamma_{opt}}) \rvert C \lvert \psi(\vec{\beta_{opt}}, \vec{\gamma_{opt}}) \rangle \). Este valor aproximará \(\min_{x}f(x)\)
\end{enumerate}

Para el estado construido se cumple que para un número de capas \(p \rightarrow \infty\):

\begin{align*}
  \min_{\beta, \gamma} \langle \psi(\vec{\beta}, \vec{\gamma}) \rvert C \lvert \psi(\vec{\beta}, \vec{\gamma}) \rangle = \min_{x} f(x)
\end{align*}

Esto es equivalente a decir que el estado fundamental del sistema corresponde al valor mínimo de la función de coste.  % TODO: \ref{sec:6-apendice-estado_fundamental}

De esta forma, el optimizador clásico llamará en repetidas iteraciones a una función \textit{execute\_circuit} con una entrada \((\vec{\beta}, \vec{\gamma})\) que devolverá como salida un valor mayor o igual al mínimo de la función de coste. Este resultado es el que el optimizador debe minimizar.

En este caso el optimizador clásico que se utilizará es COBYLA (Constrained Optimization BY Linear Approximation), de la librería de \textit{Python} \textit{scipy.minimize}.  % TODOO: Citar de alguna fuente

%%% Local Variables:
%%% mode: latex
%%% TeX-master: "../tfgtfmthesisuam"
%%% End:
