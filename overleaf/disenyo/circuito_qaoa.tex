El sistema cuántico en QAOA se implementa en un espacio de Hilbert (sección~\ref{sec:8-concepto-postulados}) de $2^n$ dimensiones, donde \textit{n} es el número de bits de entrada en la función de coste clásica. Esto quiere decir que se tendrán tantos qubits como bits tenga la entrada de la función de coste $f(x)$.
% \\
% La base computacional se representa como $\{\ket{x} : x \in {\{0, 1\}}^n\}$.

La idea general de QAOA se basa en preparar un estado parametrizado \(\lvert \psi(\vec{\beta}, \vec{\gamma})\rangle\) tal que, con los valores adecuados \( (\vec{\beta_{opt}}, \vec{\gamma_{opt}}) \), el estado \(\lvert\psi(\vec{\beta_{opt}}, \vec{\gamma_{opt}})\rangle\) encuentre la solución al problema. Los parámetros de dicho estado son los vectores:

\begin{align}
  \vec{\beta} = [\beta_0, \beta_1, \ldots , \beta_{p-1}], \vec{\gamma} = [\gamma_0, \gamma_1, \ldots , \gamma_{p-1}]
\end{align}

Donde \textit{p} es el número de capas del circuito y $\beta_i, \gamma_i \in{[0, 2\pi]}$.

En este estado se diferencian un estado inicial ($\ket{\psi_0}$) y dos operadores, mencionados en la \textit{sección~\ref{sec:2-qaoa}}, el hamiltoniano del problema y el hamiltoniano de mezcla.
Estos se combinan de la siguiente forma:

\begin{align}
  \lvert \psi(\vec{\beta}, \vec{\gamma})\rangle = U(B, \beta_{p-1})U(C, \gamma_{p-1})U(B, \beta_{p-2})U(C, \gamma_{p-2}) \ldots U(B, \beta_0)U(C, \gamma_0) \ket{\psi_0}
\end{align}

El aumento del número de capas ($p$) mejora el resultado del algoritmo, pero también aumenta la profundidad del circuito, por lo que lo hace más susceptible de sufrir por el ruido de la ejecución.

\subsection{Estado inicial}
El estado inicial del sistema es la superposición equiprobable, definida como

\begin{align}
  \ket{\psi_0} = \frac{1}{\sqrt{2^n}} \sum_x\ket{x}
  = {(\frac{1}{\sqrt{2}} (\ket{0} + \ket{1}))}^{\otimes n}
  = H^{\otimes n} \ket{0}^{\otimes n}
\end{align}

Este estado inicial se construye añadiendo operadores de Hadamard a \textit{n} qubits inicializados a $\ket{0}$, lo que genera un estado equiprobable, es decir, donde la probabilidad de obtener una cadena cualquiera en ${\{0, 1\}}^n$ al medir el estado sería en cualquier caso $\frac{1}{2^n}$.

\subsection{Hamiltoniano de mezcla y del problema\label{sec:3-problem_y_mixer_hamiltonian}}

Estos dos hamiltonianos son operadores unitarios definidos como:

\begin{align}
  U(B, \beta) = e^{-i \beta B} \\
  U(C, \gamma) = e^{-i \gamma C}
\end{align}

Como ha sido explicado en la \textit{sección~\ref{sec:2-qaoa}}, el operador $U(C, \gamma)$ modifica la fase de los estados, haciendo que los de menor energía aumenten en fase y los de mayor energía disminuyan.
\\
También se ha explicado cómo el operador $U(B, \beta)$ realiza rotaciones de tal forma que aumente la amplitud, y por ende la probabilidad, de los estados con mayor fase.

\subsubsection{Operador \textit{B}}
\textit{B} se construye a partir de puertas Pauli-X ($\sigma^x$) y únicamente depende de la cantidad de qubits del sistema.
Se define de la siguiente forma:

\begin{align}
  &B = \sum_{i=0}^{n-1}\sigma^x_{i} \\
  &U(B, \beta) = e^{-i \beta B} = \prod_{j=0}^{n-1}e^{-i \beta \sigma^x_j} = \prod_{j=0}^{n-1}Rx_i(2*\beta)
\end{align}

Donde $Rx(\alpha)$ es un operador que realiza una rotación de $\alpha$ radianes sobre el eje x de la esfera de Bloch.
\\
También, cuando un operador lineal tiene un subíndice se refiere a que es aplicado a ese qubit, por lo que la expresión $Rx_i(\alpha)$ es una puerta $Rx(\alpha)$ aplicada al qubit i-ésimo.
Es equivalente a $I \otimes I \otimes \ldots \otimes R_x(\alpha) \otimes \ldots I \otimes I$.

De esta forma, la variable $\beta$ del hamiltoniano de mezcla representa un ángulo, y por lo tanto es periódico.
Esto es importante porque el aumento indefinido de $\beta$ puede llevar a disminuir la amplitud de los estados con más fase.


\subsubsection{Operador \textit{C}\label{sec:3-operador_c}}

\textit{C} debe ser un operador diagonal (\textit{sección~\ref{sec:8-concepto-diagonal}}) en la base computacional $\ket{x}$, con los valores de la función objetivo.
Esto es equivalente a tener los valores $f(0 \ldots 0), f(0 \ldots 1), \ldots, f(1 \ldots 1)$ en la diagonal de $C$ y el resto de elementos con valor 0.
Para un vector $\ket{x}$ de la base computacional se cumple:

\begin{align}
  \bra{x}C\ket{x} = f(x)
\end{align}

Para obtener este operador $C$ se parte de una función de coste en su formato QUBO, con la forma $f(x)$, donde $x=x_0x_1 \ldots x_{n-1}, x_i \in \{0, 1\}$.
\\
Un primer paso es convertir la función en formato QUBO a su formato Ising~\cite{ising_formulations_of_np_problems}.
Una función Ising es equivalente a una tipo QUBO, solo que en lugar de aceptar cadenas del conjunto ${\{0, 1\}}^n$ acepta cadenas de ${\{-1, 1\}}^n$ para su entrada.
Esto es necesario porque, como se verá más adelante, para convertir una función de coste a un hamiltoniano $C$ se deben usar operadores cuyos autovalores correspondan a los valores que pueden tomar las variables del sistema, en este caso $-1$ y $1$.
Para realizar este paso se puede realizar un cambio de variable como el siguiente:

\begin{align}
  &f(x) \rightarrow C(z) \\
  &x_i \rightarrow \frac{1 - z_i}{2} \nonumber
\end{align}

De esta forma $z_i = 1$ si $x_i = 0$ y $z_i = -1$ si $x_i = 1$.

La forma de pasar de la función $C(z)$ al operador $C$ es sustituyendo las variables $z_i$ por puertas Pauli-Z en el qubit $i$ ($\sigma^z_i$).
\\
El motivo por el que para muchos algoritmos cuánticos se utiliza el formato Ising es porque los autovalores de la puerta Pauli-Z son $-1$ y $1$, lo que permite este paso entre variables $z$ y operadores $\sigma^z$.
\\
La justificación detallada se encuentra en la \textit{sección~\ref{CAP:F_CLASICA_A_HAMILTONIANO}} del apéndice.

Como se verá en aplicaciones concretas de QAOA, al igual que el hamiltoniano $U(B, \beta)$ se expresa mediante operadores $Rx(\alpha)$, la forma de expresar el hamiltoniano $U(C, \gamma)$ es mediante puertas $Rz(\alpha)$, cuya definición es:

\begin{align}
  Rz_i(\alpha) = e^{-i \alpha \sigma^z_i}
\end{align}

De manera análoga a $Rx(\alpha)$, $Rz(\alpha)$ representa una rotación de $\alpha$ radianes sobre el eje Z de la esfera de Bloch.
\\
De esto se concluye que, al igual que en el caso del hamiltoniano de mezcla, el hamiltoniano del problema también tiene una entrada que representa un ángulo.
Por lo tanto si se aumenta lo suficiente, disminuirá la fase de los estados con menor energía, produciendo un resultado contrario al deseado.


%%% Local Variables:
%%% mode: latex
%%% TeX-master: "../tfgtfmthesisuam"
%%% End:
