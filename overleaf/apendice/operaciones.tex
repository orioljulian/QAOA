\section{Exponente de una matriz}

En esta sección se muestra el desarrollo necesario para operar sobre operador $e^{i*\theta*A}$, dados dos casos:

\subsection{A*A = I\label{sec:8-A_A_es_I}}

\begin{itemize}
\item \textbf{Definiciones:}
  \begin{align*}
    I = \begin{bmatrix}
      1 & 0 \\
      0 & 1
    \end{bmatrix}
    , A^2 = I
  \end{align*}

\item \textbf{Desarrollo:}

  \begin{align*}
    \exp(i*\theta*A) &= \sum_{n = 0}^{\infty} \frac{{(i*\theta)}^n * A^n}{n!} = \\
                &= (1*A^0 - \frac{\theta^2*A^2}{2!} + \frac{\theta^4*A^4}{4!} - \ldots) +
                  i*(\theta*A^1 - \frac{\theta^3*A^3}{3!} + \frac{\theta^5*A^5}{5!} - \ldots) = \\
                &= I*(1 - \frac{\theta^2}{2!} + \frac{\theta^4}{4!} - \ldots) +
                  i*A*(\theta - \frac{\theta^3}{3!} + \frac{\theta^5}{5!} - \ldots)
                  = \cos(\theta)*I + i*\sin(\theta)*A
  \end{align*}
\end{itemize}

\subsection{A es diagonalizable\label{sec:8-funcion A diagonalizable}}

Un operador $A$ sobre un espacio $V$ es diagonalizable si es diagonal con respecto a alguna base ortonormal de $V$. La representación diagonal de $A$ tiene la siguiente forma:

\begin{align*}
  A = \sum_\lambda \lambda \ket{\lambda}\bra{\lambda}
\end{align*}
Donde $\lambda$ son los autovalores de $A$ y $\ket{\lambda}$ sus respectivos autovectores.

Dada una función $f$ definida sobre los números complejos se puede definir una función correspondiente sobre $A$ utilizando su representación diagonal:

\begin{align*}
  f(A) = \sum_\lambda f(\lambda) \ket{\lambda}\bra{\lambda}
\end{align*}

Concretamente, para $e^{i \theta A}$:

\begin{align*}
   \exp( -i \theta A) = \sum_\lambda \exp(\lambda) \ket{\lambda}\bra{\lambda}
\end{align*}

%%% Local Variables:
%%% mode: latex
%%% TeX-master: "../tfgtfmthesisuam"
%%% End:
