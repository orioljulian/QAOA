En esta sección se muestra una breve documentación sobre la ejecución de programas tanto en Qiskit como en D-Wave.

\section{Qiskit}

Las librerías de Qiskit parecen encontrarse todavía en fases tempranas de su desarrollo, por lo que siguen sufriendo cambios en los métodos para ejecutar circuitos.
Por esto es necesario utilizar las mismas versiones de Qiskit, ya que en otro caso podría haber discrepancia en las funciones.

\begin{table}{}{Versiones de Python para Qiskit}
  \centering
  \begin{tabular}{|c|c|}
    \hline
    \textbf{Paquete}    & \textbf{Versión} \\ \hline
    qiskit              & 0.45.1           \\
    qiskit-ibm-runtime  & 0.17.0           \\
    qiskit-ibm-provider & 0.8.0            \\
    qiskit-aer          & 0.13.2           \\
    qiskit-terra        & 0.45.1           \\
    \hline
  \end{tabular}
\end{table}

Concretamente se emplea el entorno de \textit{Qiskit Runtime}, que es el más utilizado actualmente.

\paragraph{Ejemplo de ejecución en simulador}

\begin{lstlisting}[language=Python,label=cod:8-ejecucion_qiskit_simulador,caption={Ejemplo de ejecución en simulador de Qiskit},style=numbered]
from qiskit import QuantumCircuit
from qiskit.primitives import Sampler
from qiskit.visualization import plot_histogram

shots = 1024
sampler = Sampler(options={"shots": shots})

# Construccion de circuito con 2 qubits
qc = QuantumCircuit(2)
qc.h(0)           # Aplicar puerta Hadamard al qubit 0
qc.cnot(0, 1)     # Aplicar puerta X con 0 como qubit de control y 1 como objetivo
qc.measure_all()  # Instrucción para medir

result = sampler.run(qc).result()  # Ejecutar circuito

probabilities = result.quasi_dists[0]  # Resultados obtenidos en la ejecución
# {0: 0.51, 3: 0.49}

quasi_probabilities = probabilities.binary_probabilities()
# {00: 0.51, 11: 0.49}

plot_histogram(quasi_probabilities)
\end{lstlisting}

Como primer paso un circuito debe ser construido.
Una vez especificada la cantidad de qubits (línea 9 del \textit{código~\ref{cod:8-ejecucion_qiskit_simulador}}) a incluir se aplican puertas cuánticas explícitamente (líneas 10{-}12 del \textit{código~\ref{cod:8-ejecucion_qiskit_simulador}}).

Para ejecutar en un simulador existen las llamadas primitivas \textit{qiskit.primitives.Sampler} y \textit{qiskit.primitives.Estimator}, que no requieren de la conexión a un computador cuántico real para ser ejecutados:
\begin{itemize}
\item \textit{Sampler}: Utilizado para ejecutar el circuito cuántico un número (\textit{shots}) de veces y mostrar los resultados.
  Esto es equivalente a decir que se calcula $C\ket{0 \ldots 0}$, donde $C$ es equivalente a \textit{qc} en el código.
  
\item \textit{Estimator}: Calcula el valor esperado de un circuito parametrizado (ansatz) y un hamiltoniano (denominado observable).
  Como ya ha sido explicado en la definición de QAOA, el valor esperado se calcula como:

  \begin{align}
    \langle \psi(\theta) \rvert C \lvert \psi(\theta) \rangle
  \end{align}

  Donde $\ket{\psi(\theta)}$ es el estado final del circuito parametrizado y $C$ es el hamiltoniano.

  En el trabajo este objeto no ha sido utilizado, ya que el valor esperado necesario en QAOA (\textit{paso~\ref{it:3-algoritmo_valor_esperado}} del algoritmo) se puede calcular de manera clásica, a partir de la función de coste.
\end{itemize}

El circuito del \textit{código~\ref{cod:8-ejecucion_qiskit_simulador}} se ejecuta inicializando un objeto \textit{Sampler} con los parámetros necesarios, como el número de muestras (\textit{shots}).
La ejecución se realiza en la línea 16 y el resto de instrucciones son para obtener los datos en un diccionario ``\{bits: probabilidades\}'' para después obtener su histograma con \textit{qiskit.visualization.plot\_histogram}.


\paragraph{Ejemplo de ejecución en ordenador real}

\begin{lstlisting}[language=Python,label=cod:8-ejecucion_qiskit_real,caption={Ejemplo de ejecución en computador real de Qiskit},style=numbered]
from qiskit_ibm_runtime import QiskitRuntimeService
from qiskit_ibm_runtime import QiskitRuntimeService, Sampler, Session, Options

# Guardar token asociado a la cuenta de Qiskit
# Solo se tiene que realizar una vez en cada máquina
QiskitRuntimeService.save_account(channel="ibm_quantum", token="token_ejemplo",
                                    overwrite=True)

QiskitRuntimeService.active_account(QiskitRuntimeService(channel="ibm_quantum"))
service = QiskitRuntimeService(channel="ibm_quantum")

quantum_computer = "ibmq_qasm_simulator"     # Nombre del ordenador al que conectarse
backend = service.backend(quantum_computer)  # Conexión a dicho ordenador

options = Options()  # Objeto opciones para Sampler y Estimator
options.execution.init_qubits = True  # Inicializar qubits a 0
options.execution.shots = 512  # Número de muestras

qc = QuantumCircuit(2)
qc.h(0)
qc.cnot(0, 1)
qc.measure_all()

with Session(backend=backend, service=service) as session:
    sampler = Sampler(session=session, options=options)
    quasi_probabilities = sampler.run(qc).result().quasi_dists[0].binary_probabilities()
    quasi_probabilities  # Resultado
\end{lstlisting}

Para poder realizar una conexión con un ordenador cuántico se debe introducir el token de \textit{Qiskit} generado al crear una cuenta.
Esto se hace con la función \textit{qiskit\_ibm\_runtime.QiskitRuntimeService.save\_account()}, que en Linux genera un fichero \path{~/.qiskit/qiskit-ibm.json}.
\\
Tanto para la ejecución en un simulador como en un ordenador real se pueden usar unos objetos \textit{Sampler} y \textit{Estimator} equivalentes, solo que para el caso del simulador se obtienen de la librería \textit{qiskit.primitives} y en el caso de conectarse a un computador se obtienen de \textit{qiskit\_ibm\_runtime}.
\\
Tras conectarse a un ordenador concreto con \textit{qiskit\_ibm\_runtime.QiskitRuntimeService.backend()} se debe crear un entorno \textit{qiskit\_ibm\_runtime.Session()}, dentro del cual las ejecuciones se realizan de la forma habitual.


\section{D-Wave}

Los paquetes utilizados son los siguientes:

\begin{table}{}{Versiones de Python para D-Wave}
  \centering
  \begin{tabular}{|c|c|}
    \hline
    \textbf{Paquete} & \textbf{Versión} \\ \hline
    dwave-ocean-sdk  & 6.9.0            \\
    \hline
  \end{tabular}
\end{table}

\begin{lstlisting}[language=Python,label=cod:8-ejecucion_qiskit,caption={Ejemplo de ejecución en D-Wave},style=numbered]
from dimod import BinaryQuadraticModel
# Inicializar modelo
bqm = BinaryQuadraticModel("BINARY")

# Función de coste f(x, y) = 2x + 3y
peso = {"x": 2, "y": 3}
for variable in ["x", "y"]:
    bqm.add_variable(variable, peso[variable])

# Añadir restriccion: 3x + 2y - 9 = 0
bqm.add_linear_equality_constraint([("x", 3), ("y", 2)], constant=-9, lagrange_multiplier=P)

# Ejecutar programa
sampler = EmbeddingComposite(DWaveSampler())
sampleset = sampler.sample(bqm, num_reads=1024)
print(sampleset)
\end{lstlisting}

En D-Wave el objeto \textit{dimod.BinaryQuadraticModel} (línea 3 del \textit{código~\ref{cod:8-ejecucion_qiskit}}) describe un modelo que sirve para resolver funciones de coste en versión QBO o Ising con restricciones.
En este caso la inicialización a ``BINARY'' quiere decir que es tipo QBO, en el otro caso sería ``SPIN''.

Tanto la introducción de la función de coste como la introducción de restricciones se realizan de formas similares a las líneas 8 y 11 del \textit{código~\ref{cod:8-ejecucion_qiskit}}.
El argumento \textit{lagrange\_multiplier} corresponde al modificador de Lagrange, utilizado para pasar de modelo QBO a QUBO\@.

Las sentencias posteriores (líneas 14, 15 del \textit{código~\ref{cod:8-ejecucion_qiskit}}) ejecutan el modelo creado en un ordenador de D-Wave, con una cantidad de muestras \textit{num\_reads}.


%%% Local Variables:
%%% mode: latex
%%% TeX-master: "../tfgtfmthesisuam"
%%% End:
