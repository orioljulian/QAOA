En esta sección se muestra un ejemplo práctico simple, para facilitar la comprensión de la función de los dos operadores de QAOA\@.

\section{Ejemplo de modificación de fases\label{sec:8-ejemplo de modificacion de fases}}

En un espacio de Hilbert de $2^2$ dimensiones, dado un hamiltoniano $H$ definido de la siguiente forma:

\begin{align*}
  H = \begin{bmatrix}
    3 & 0 & 0 & 0 \\
    0 & 1 & 0 & 0 \\
    0 & 0 & 2 & 0 \\
    0 & 0 & 0 & 4
  \end{bmatrix} = 3*\ket{00}\bra{00} + 1*\ket{01}\bra{01} + 2*\ket{10}\bra{10} + 4*\ket{11}\bra{11}
\end{align*}

Como $H$ es diagonal en la base computacional se puede calcular $U(H, \gamma)$ como se explica en la \textit{sección~\ref{eq:8-funcion A diagonalizable}}

El estado de mínima energía de $H$ sería $\min_{x \in {\{0, 1\}}^2}{\bra{x}H\ket{x}}$, por lo que el estado es $\ket{x} = \ket{01}$, con energía 1.
De esta forma, el \textit{hamiltoniano del problema} sería $U(H, \gamma) = e^{-i \gamma H}$. Tomando un valor de $\gamma = 0.1$, el resultado de aplicarlo a un estado en superposición equiprobable $\ket{\psi}$ es el siguiente:

\begin{align*}
  &\ket{\psi} = \frac{1}{2}*(\ket{00} + \ket{01} + \ket{10} + \ket{11}) \\
  &\gamma = 0.1 \\
  &U(H, \gamma) \ket{\psi} =
    \begin{bmatrix}
      e^{-i * 0.1 * 3} & 0                & 0                & 0 \\
      0                & e^{-i * 0.1 * 1} & 0                & 0 \\
      0                & 0                & e^{-i * 0.1 * 2} & 0 \\
      0                & 0                & 0                & e^{-i * 0.1 * 4}
    \end{bmatrix}
    *\frac{1}{2}
    \begin{bmatrix}
      1 \\
      1 \\
      1 \\
      1
    \end{bmatrix} = \frac{1}{2}
    \begin{bmatrix}
      e^{-i * 0.3} \\
      e^{-i * 0.1} \\
      e^{-i * 0.2} \\
      e^{-i * 0.4}
    \end{bmatrix}
\end{align*}

Como se puede ver las fases relativas  % TODO: Citar explicación de fases relativas?
en $\ket{\psi}$ son de 0 para todos los valores, mientras que para $U(H, \gamma) \ket{\psi}$ el estado de mayor fase relativa es $\ket{01}$.

También es importante recalcar que las probabilidades de medición de los valores no han sido modificadas. Es el caso de 01:

\begin{align*}
  P_{\psi}(01) = \bra{\psi}M_{01}^\dagger M_{01}\ket{\psi} = \bra{\psi}M_{01}\ket{\psi} = |\frac{1}{2}*e^{-0.1i}|^2 = \frac{1}{4}
\end{align*}

Donde $M_{01} = \bra{01}\ket{01}$ es un operador de medición para el valor 01.
Se han tomado las definiciones del postulado de la medición explicado en~\cite{Nielsen_Chuang_2010}.


\section{Ejemplo de modificación de probabilidades}
Tomando el resultado de la \textit{sección~\ref{sec:8-ejemplo de modificacion de fases}} se puede ver cómo el \textit{hamiltoniano del problema} aumenta la probabilidad de medición de 01.

\begin{align*}
  &\sigma^x = \begin{bmatrix}
    0 & 1 \\
    1 & 0
  \end{bmatrix}, I = \begin{bmatrix}
    1 & 0 \\
    0 & 1
  \end{bmatrix} \\
  &\beta = 0.1 \\
  &B = \sigma^x \otimes I + I \otimes \sigma^x = \begin{bmatrix}
    0 & 1 & 1 & 0 \\
    1 & 0 & 0 & 1 \\
    1 & 0 & 0 & 1 \\
    0 & 1 & 1 & 0
  \end{bmatrix} \\
  &U(B, \beta) = e^{-i \beta B} = \begin{bmatrix}
    {\cos(\beta)}^2       & -i*\sin(\beta)\cos(\beta) & -i*\sin(\beta)\cos(\beta) & -{\sin(\beta)}^2      \\
    -i*\sin(\beta)\cos(\beta) & {\cos(\beta)}^2       & -{\sin(\beta)}^2      & -i*\sin(\beta)\cos(\beta) \\
    -i*\sin(\beta)\cos(\beta) & -{\sin(\beta)}^2      & {\cos(\beta)}^2       & -i*\sin(\beta)\cos(\beta) \\
    -{\sin(\beta)}^2      & -i*\sin(\beta)\cos(\beta) & -i*\sin(\beta)\cos(\beta) & {\cos(\beta)}^2
  \end{bmatrix} = \\
  &= \begin{bmatrix}
    0.99     & -0.0993j & -0.0993j & -0.01    \\
    -0.0993j & 0.99     & -0.01    & -0.0993j \\
    -0.0993j & -0.01    & 0.99     & -0.0993j \\
    -0.01    & -0.0993j & -0.0993j & 0.99     \\
  \end{bmatrix} \\
  &U(B, \beta)* \frac{1}{2} \begin{bmatrix}
    e^{-i * 0.3} \\
    e^{-i * 0.1} \\
    e^{-i * 0.2} \\
    e^{-i * 0.4}
  \end{bmatrix} = \begin{bmatrix}
    0.4535 -0.2424i \\
    0.4536 -0.1416i \\
    0.4462 -0.191i  \\
    0.4364 -0.2894i  \\
    \end{bmatrix}
\end{align*}

Donde $\sigma^x$ es conocida como puerta Pauli-X, que equivale a una rotación en el eje X de la esfera de Bloch.


%%% Local Variables:
%%% mode: latex
%%% TeX-master: "../tfgtfmthesisuam"
%%% End:
