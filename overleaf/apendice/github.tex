Se ha habilitado un repositorio de Github\cite{codigo_tfg} con todo el contenido de este TFG, incluidos los cuadernos de Jupyter utilizados para realizar las ejecuciones de los algoritmos de este trabajo.

Para ambos Qiskit y D-Wave existen más cuadernos con problemas resueltos y modificaciones de los problemas presentados que los mostrados en el trabajo de fin de grado.

\paragraph{Ejecuciones en D-Wave}

Se ha proporcionado un fichero \path{requirements_dwave.txt} con los paquetes necesarios para realizar las ejecuciones de \textit{quantum annealing}.
\\
Para realizar estas ejecuciones además es necesario crear una cuenta en \url{https://www.dwavesys.com}, para tener un token asociado con el que poder conectarse a los sistemas de D-Wave

El código ejecutado se encuentra en el subdirectorio \path{codigo/dwave}, separado en cuadernos según el problema a resolver (\path{shortest-path-dwave.ipynb} y \path{max-cut.ipynb}).

\paragraph{Ejecuciones en Qiskit}

Se ha proporcionado un fichero \path{requirements_qiskit.txt} con los paquetes necesarios para realizar ejecuciones de QAOA\@.
También es importante realizar las ejecuciones en un entorno con Python $\le$ 3.8 (solo ha sido probado para Python 3.8), ya que para versiones superiores existen errores en las librerías de Qiskit.
\\
Para realizar ejecuciones en ordenadores reales se debe generar un token de Qiskit en \url{https://quantum.ibm.com}, pero para cualquier otra ejecución en simulador no es necesario, ya que pueden ser ejecutadas en la máquina local.

Para las ejecuciones en Qiskit se han utilizado los siguientes cuadernos, todos presentes en el subdirectorio \path{codigo/}:

\begin{itemize}
\item \path{print_circuits.ipynb}: Utilizado para obtener las imágenes de los circuitos presentadas en este trabajo.
  Se ha separado de otras secciones porque por defecto un circuito cuántico no muestra los parámetros $\gamma$ y $\beta$, sino coeficientes ya resueltos (es decir, para $\gamma = 1.5$ en un operador $Rz(\gamma*2)$ mostraría $Rz(3)$).

\item \path{grafo-generico/shortest_path_qaoa.ipynb} y \path{max-cut-qiskit/max_cut_qaoa.ipynb}: Estos dos cuadernos contienen la implementación de QAOA, generación de estadísticas y ejecuciones en ordenadores reales.
  
  Ha sido separado en dos ficheros distintos, según el tipo de problema, para diferenciar las modificaciones en las pruebas y los dibujos de grafos, pero las ejecuciones en circuitos cuánticos y generación de estadísticas son idénticas.
  
\item \path{grafo-generico/QAOAAnsatz.ipynb} y \path{max-cut-qiskit/qiskit-QAOAAnsatz.ipynb}: De manera análoga a los ficheros anteriores, contienen las ejecuciones utilizando la implementación de Qiskit de QAOA, separado según el tipo de problema a resolver.
\end{itemize}


%%% Local Variables:
%%% mode: latex
%%% TeX-master: "../tfgtfmthesisuam"
%%% End:
