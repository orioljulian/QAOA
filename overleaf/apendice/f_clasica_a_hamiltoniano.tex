Esta sección está destinada a demostrar el paso de una función clásica $f(x)$ en su versión QUBO a su hamiltoniano correspondiente $C$.

\begin{itemize}
\item Definiciones:
  \begin{itemize}
  \item $f(x)$: Función clásica en formato QUBO (\textit{Quadratic Unconstrained Binary Optimization}), donde $x \in {\{0 , 1\}}^n$.

  \item $C(z)$: Función clásica en formato Ising equivalente a $f(x)$, pero donde $z \in {\{-1 , 1\}}^n$.

  \item $\sigma_z$: Operador Pauli-Z.

    \begin{align*}
      \sigma^z = \begin{bmatrix}
        1 & 0 \\
        0 & -1
      \end{bmatrix}
    \end{align*}
  \end{itemize}
\end{itemize}

Concretamente se pretende demostrar que sustituyendo las variables $z_i$ en $C(z)$ por puertas $\sigma^z$ en el qubit \textit{i} se puede obtener el operador $C$ tal que se cumpla:

\begin{align*}
  f(x) = \bra{x} C \ket{x}, \forall{x \in {\{0, 1\}}^n}
\end{align*}

Esto es equivalente a decir que el operador $C$ es diagonal, con valores correspondientes a la función clásica $f(x)$:

\begin{align*}
  &C \ket{x} = \begin{bmatrix}
    C(0 \ldots 00) & 0          & 0  & 0 \\
    0          & C(0 \ldots 01) & 0  & 0 \\
    \vdots         &            & \ddots &   \\
    0          & 0          & 0  & C(1 \ldots 11)
  \end{bmatrix} \cdot
  \begin{bmatrix}
    0  \\
    \vdots \\
    1  \\
    \vdots \\
    0
  \end{bmatrix} = f(x) \ket{x} & \forall x \in {\{0, 1\}}^n
\end{align*}

Los autovalores de $\sigma^z$ son $-1, +1$, y sus respectivos autovectores la base computacional:

\begin{align*}
  \sigma^z\ket{0} = \begin{bmatrix} 1 & 0 \\ 0 & -1 \end{bmatrix} \cdot \begin{bmatrix} 1 \\ 0 \end{bmatrix} = +1*\ket{0} \\
  \sigma^z\ket{1} = \begin{bmatrix} 1 & 0 \\ 0 & -1 \end{bmatrix} \cdot \begin{bmatrix} 0 \\ 1 \end{bmatrix} = -1*\ket{1}
\end{align*}

Esto puede ser generalizado como:

\begin{align*}
  \sigma^z\ket{x} = {(-1)}^x\ket{x}, x \in \{0, 1\}
\end{align*}

En las funciones de coste utilizadas existen dos términos distintos en los que pueden aparecer variables $z_i$:
\begin{itemize}
\item $c*z_i$ \\
  Un operador $\sigma^z$ actuando en el qubit \textit{i} se desarrolla de esta forma:

  \begin{align}\label{eq:8-demostracion-z_i}
    &\sigma^z_i \ket{x_0 \ldots x_{n-1}} = I \otimes \ldots \otimes \sigma^z_i \otimes \ldots \otimes I \ket{x_0 \ldots x_{n-1}} = {(-1)}^{x_i} \ket{x_0 \ldots x_{n-1}}     & x_i \in \{0, 1\}, i \in 1, \ldots n
  \end{align}
  
\item $c*z_j*z_k, j \ne k$ \\
  De forma similar al caso anterior, dos operadores $\sigma^z$ actuando en dos qubits \textit{i}, \textit{j} se desarrollan de esta forma:

  \begin{align}\label{eq:8-demostracion-z_i_j}
    \sigma^z_i \sigma^z_j \ket{x_0 \ldots x_{n-1}} &= I \otimes \ldots \otimes \sigma^z_i \otimes \sigma^z_j \otimes \ldots \otimes I \ket{x_0 \ldots x_{n-1}} = \nonumber \\
                                    &= {(-1)}^{x_i} {(-1)}^{x_j} \ket{x_0 \ldots x_{n-1}}     & x_i \in \{0, 1\}, i \in 1, \ldots n
  \end{align}

\item $c*z_i*z_i = c$ \\
  Porque $z_i \in \{-1, 1\} \forall i \in 0, \ldots n - 1$

\end{itemize}

Esto permite que se realice un cambio de variable a la función $C(z)$ tal que $z_i \rightarrow {(-1)}^{x_i}$, donde se cumple que $x_i \in \{0, 1\}$ teniendo en cuenta que $z_i \in \{-1, 1\}$.

Dado este cambio de variable, a su vez se podrán utilizar las ecuaciones \textit{\ref{eq:8-demostracion-z_i}} y \textit{\ref{eq:8-demostracion-z_i_j}} para sustituir esos términos ${(-1)}^{x_i}$ por $\sigma^z_i$


%%% Local Variables:
%%% mode: latex
%%% TeX-master: "../tfgtfmthesisuam"
%%% End:
