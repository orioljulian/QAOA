La computación cuántica adiabática (\textbf{AQC}) es, en contraposición con un enfoque orientado a circuitos, un paradigma basado en el teorema adiabático.  % TODO: Citar definición del teorema adiabático??

Sea un sistema cuántico dado por un hamiltoniano dependiente del tiempo $H(t) = (1-\frac{t}{T})H_s + \frac{t}{T}H_c$, siendo $H_s$ un hamiltoniano con un estado fundamental conocido, $H_c$ un hamiltoniano cuyo estado fundamental se quiere conocer y $T$ el tiempo total.

En este caso el teorema adiabático afirma que si el estado inicial es el estado fundamental de $H_s$ y se avanza el tiempo lentamente el sistema se mantendrá en el estado fundamental de $H(\alpha)$. Esto significa que al llegar a $t=T$ el sistema se encontrará en el estado fundamental de $H_c$.

El paradigma seguido por los ordenadores de D-Wave, Quantum Annealing, es una implementación de AQC.

% TODO: Comparación entre AQC y QAOA: https://www.mustythoughts.com/quantum-approximate-optimization-algorithm-explained

%%% Local Variables:
%%% mode: latex
%%% TeX-master: "../tfgtfmthesisuam"
%%% End:
