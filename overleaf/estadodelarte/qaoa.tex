Este algoritmo es utilizado para resolver problemas de optimización en formato Ising.  % TODO: Citar
Para esto, se construye un hamiltoniano cuyo estado de mínima energía corresponda con el mínimo de la función de optimización.

Dada una función de coste $C(x)$, un hamiltoniano $H$ y un espacio de resultados $2^{n}$ esta igualdad se describe de la siguiente forma:

\begin{align*}
  \forall{x \in 2^{n}}, C(x) = \bra{x}H\ket{x}
\end{align*}

Como este operador H no es necesariamente unitario se debe utilizar el operador $e^{iH}$,  % TODO: Explicarlo de alguna forma
en el que en lugar de encontrar el estado de menor energía se debe encontrar el estado de menor fase.  % TODO: No sé si está bien explicado

Para esto, en el algoritmo QAOA se utilizan dos operadores unitarios.
\begin{itemize}
\item El primero es el \textbf{problem hamiltonian}, $U(C, \gamma) = e^{-i \gamma H}$, correspondiente al operador previamente mencionado.

  Dado un estado con una serie de valores posibles (correspondientes a los estados de la base computacional) este operador separa las fases relativas de dichos valores, de tal forma que la fase de los valores con menor coste disminuye y la de los valores con mayor coste aumenta. Esto no modifica la probabilidad de medición, ya que modificar la fase relativa manteniendo el eje de medición no modifica la probabilidad de medición de estos valores.

\item Para que estos cambios en las fases relativas se traduzcan en un aumento de la probabilidad de medición de los valores deseados se utiliza el operador \textbf{mixer hamiltonian}, $U(B, \beta) = e^{-i \beta B}$.

  Este operador realiza una rotación en el estado, de tal forma que los valores con menor fase son más probables de observar que los valores con mayor fase.

\end{itemize}

\subsection{Ejemplo de modificación de fases}
\label{sec:2-ejemplo de modificacion de fases}

En un espacio de Hilbert de $2^2$ dimensiones, dado un hamiltoniano $H$ definido de la siguiente forma:

\begin{align*}
  H = \begin{pmatrix}
    3 & 0 & 0 & 0 \\
    0 & 1 & 0 & 0 \\
    0 & 0 & 2 & 0 \\
    0 & 0 & 0 & 4
  \end{pmatrix} = 3*\ket{00}\bra{00} + 1*\ket{01}\bra{01} + 2*\ket{10}\bra{10} + 4*\ket{11}\bra{11}
\end{align*}

Como $H$ es diagonal en la base computacional se puede calcular $U(H, \gamma)$ como se explica en la sección \ref{eq:8-funcion A diagonalizable}

El estado de mínima energía de $H$ sería $\min_{x \in \{0, 1\}^2}{\bra{x}H\ket{x}}$, por lo que el estado es $\ket{x} = \ket{01}$, con energía 1.
De esta forma, el \textbf{problem hamiltonian} sería $U(H, \gamma) = e^{-i \gamma H}$. Tomando un valor de $\gamma = 0.1$, el resultado de aplicarlo a un estado en superposición equiprobable $\ket{\psi}$ es el siguiente:

\begin{align*}
  &\ket{\psi} = \frac{1}{2}*(\ket{00} + \ket{01} + \ket{10} + \ket{11}) &&\\
  &\gamma = 0.1 \\
  &U(H, \gamma) \ket{\psi} =
    \begin{pmatrix}
      e^{-i * 0.1 * 3} & 0                & 0                & 0 \\
      0                & e^{-i * 0.1 * 1} & 0                & 0 \\
      0                & 0                & e^{-i * 0.1 * 2} & 0 \\
      0                & 0                & 0                & e^{-i * 0.1 * 4}
    \end{pmatrix}
    *\frac{1}{2}
    \begin{pmatrix}
      1 \\
      1 \\
      1 \\
      1
    \end{pmatrix} = \frac{1}{2}
    \begin{pmatrix}
      e^{-i * 0.3} \\
      e^{-i * 0.1} \\
      e^{-i * 0.2} \\
      e^{-i * 0.4}
    \end{pmatrix}
\end{align*}

Como se puede ver las fases relativas  % TODO: Citar explicación de fases relativas?
en $\ket{\psi}$ son de 0 para todos los valores, mientras que para $U(H, \gamma) \ket{\psi}$ el estado de mayor fase relativa es $\ket{01}$.

También es importante recalcar que las probabilidades de medición de los valores no han sido modificadas. Es el caso de 01:

\begin{align*}
  P_{\psi}(01) = \bra{\psi}M_{01}^\dagger M_{01}\ket{\psi} = \bra{\psi}M_{01}\ket{\psi} = |\frac{1}{2}*e^{-0.1i}|^2 = \frac{1}{4}
\end{align*}

Donde $M_{01} = \bra{01}\ket{01}$ es un operador de medición para el valor 01.
Se han tomado las definiciones del postulado de la medición explicado en \cite{Nielsen_Chuang_2010}.


\subsection{Ejemplo de modificación de probabilidades}
Tomando el resultado de la sección \ref{sec:2-ejemplo de modificacion de fases} se puede ver cómo el operador \textbf{mixer hamiltonian} aumenta la probabilidad de medición de 01.

\begin{align*}
  &\sigma^x = \begin{pmatrix}
    0 & 1 \\
    1 & 0
  \end{pmatrix}, I = \begin{pmatrix}
    1 & 0 \\
    0 & 1
  \end{pmatrix} \\
  &\beta = 0.1 \\
  &B = \sigma^x \otimes I + I \otimes \sigma^x = \begin{pmatrix}
    0 & 1 & 1 & 0 \\
    1 & 0 & 0 & 1 \\
    1 & 0 & 0 & 1 \\
    0 & 1 & 1 & 0
  \end{pmatrix} \\
  &U(B, \beta) = e^{-i \beta B} = \begin{pmatrix}
    cos(\beta)^2        & -i*sin(\beta)cos(\beta) & -i*sin(\beta)cos(\beta) & -sin(\beta)^2       \\
    -i*sin(\beta)cos(\beta) & cos(\beta)^2        & -sin(\beta)^2       & -i*sin(\beta)cos(\beta) \\
    -i*sin(\beta)cos(\beta) & -sin(\beta)^2       & cos(\beta)^2        & -i*sin(\beta)cos(\beta) \\
    -sin(\beta)^2       & -i*sin(\beta)cos(\beta) & -i*sin(\beta)cos(\beta) & cos(\beta)^2
  \end{pmatrix} = \\
  &= \begin{pmatrix}
    0.99     & -0.0993j & -0.0993j & -0.01    \\
    -0.0993j & 0.99     & -0.01    & -0.0993j \\
    -0.0993j & -0.01    & 0.99     & -0.0993j \\
    -0.01    & -0.0993j & -0.0993j & 0.99     \\
  \end{pmatrix} \\
  &U(B, \beta)* \frac{1}{2} \begin{pmatrix}
    e^{-i * 0.3} \\
    e^{-i * 0.1} \\
    e^{-i * 0.2} \\
    e^{-i * 0.4}
  \end{pmatrix} = \begin{pmatrix}
    0.4535 -0.2424i \\
    0.4536 -0.1416i \\
    0.4462 -0.191i  \\
    0.4364 -0.2894i  \\
    \end{pmatrix}
\end{align*}

Donde $\sigma^x$ es conocida como puerta Pauli-X, que equivale a una rotación en el eje X de la esfera de Bloch.

% TODO: Hablar sobre la resolución de $e^{-i \alpha H}$ de la ecuación de Schrodinger

%%% Local Variables:
%%% mode: latex
%%% TeX-master: "../tfgtfmthesisuam"
%%% End:
