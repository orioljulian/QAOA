\documentclass[epsbased,copyright,final,printable,covers,extendedindex,firstnumbered,tfg,gnuplot]{tfgtfmthesisuam}
\usepackage{needspace}
\usepackage{mdwlist}
\usepackage{amsmath}
\usepackage{physics}
\usepackage{enumitem}
\usepackage{graphicx}
\usepackage{subfigure}
\setlist[enumerate]{label*=\arabic*.}

\advisor{Francisco Gómez Arribas}

\levelin{Ingeniería Informática}
\title{Estudio de la aplicación de algoritmos cuánticos de optimización con las tecnologías cuánticas actuales}
\author{Oriol Julián Posada}
\privateaddress{C\textbackslash\ Francisco Tomás y Valiente Nº 11}
\copyrightdate{3 de Noviembre de 2017}

\datadir{data}
\graphicspath{{./img}}
\graphicsdir{img}
\logosdir{img}
\codesdir{codes}

\resumenfile{inicio/resumen}
\abstractfile{inicio/abstract}

\keywords{Algunas}
\palabrasclave{Otras}

\coverdata
{
  Escuela Politécnica Superior \\
  Universidad Autónoma de Madrid \\
  C\textbackslash Francisco Tomás y Valiente nº 11
}

\begin{document}

\chapter{Introducción\label{cap:introduccion}}{introduccion/introduccion}

\chapter{Estado del arte\label{CAP:ESTADODELARTE}}{estadodelarte/estadodelarte}

\chapter{Diseño\label{CAP:DISEÑO}}{disenyo/disenyo}

\chapter{Desarrollo\label{CAP:DESARROLLO}}{desarrollo/desarrollo}

\chapter{Integración, pruebas y resultados\label{CAP:RESULTADOS}}{resultados/resultados}

\chapter{Conclusiones y trabajo futuro\label{CAP:CONCLUSIONES}}{conclusiones/conclusiones}

%%%%%%%%%%%%%%%%%%%%%%%%%%%%%%%%%%%%%%%%%%%%%%%%%%%%%%%%%%%%%%%%%%%%%%%%%%%%%%%%


%\chapter{Bibliografía\label{CAP:bibliografia}}{bibliografia}
\newpage
\phantomsection

\addcontentsline{toc}{chapter}{Bibliografía}
\bibliographystyle{IEEEtran}
\bibliography{bibliografia}
% \chapter[Glosario, acrónimos y definiciones]{Glosario, acrónimos y definiciones\label{SEC:GLOSARIO}}{glosario}

% \appendix
% \chapter{Manual de instalación\label{CAP:INSTALACION}}{apendices/manualinstalacion}
% %  \section{Requisitos del sistema\label{SEC:REQUISITOS}}{apendices/manualinstalacion/requisitos}
%   \section{Instalación de dependencias\label{SEC:DEPENDENCIAS}}{apendices/manualinstalacion/dependencias}
% 
%     \subsection{Paquetes necesarios\label{SEC:PAQUETES}}{apendices/manualinstalacion/paquetes}
%     \subsection{Instalación\label{SEC:LIBBPF}}{apendices/manualinstalacion/libbpf}



%\chapter{Manual de usuario\label{CAP: PROGRAMADOR}}{apendices/manualusuario}
%    \section{Mostrar información del tráfico\label{SEC:MOSTRARTRAFICO}}{apendices/manualusuario/mostrartrafico}


% \chapter{Manual del programador\label{CAP: PROGRAMADOR}}{apendices/manualprogramador}
%     \section{Estructura interna de la aplicación\label{SEC:ESTRUCTURAAPlICACION}}{apendices/manualprogramador/estructuradirectorios}
%     \Needspace{50\baselineskip}
%     \section{Desarrollo interno: Aplicación XDP\label{SEC:ENTORNOC}}{apendices/manualprogramador/xdp}
%     
%        \subsection{Compilación y estructura de un programa\label{SEC:COMPILACIONXDP}}{desarrollo/compilacionxdp}
%         \subsection{Mapas BPF\label{SEC:MAPASBPF}}{desarrollo/mapasbpf}
%    \section{Cargador de eBPF\label{SEC:CARGADORXDP}}{desarrollo/cargadorxdp}
%     
%     
%     
%     \section{Desarrollo de la interfaz en Python\label{SEC:ENTORNOPYTHON}}{apendices/manualprogramador/python}
%     \section{Desarrollo externo: API REST\label{SEC:ENTORNOPYTHON}}{apendices/manualprogramador/memoria}
% 
% 
% \chapter{Diagrama de Gantt\label{CAP:GANTT}}{apendices/gantt}



% \chapter{Github, demostración y ejemplos\label{CAP:Github}}{apendices/Github}
\end{document}

%%% Local Variables:
%%% mode: latex
%%% TeX-master: t
%%% End:
