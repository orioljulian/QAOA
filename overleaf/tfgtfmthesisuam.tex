\documentclass[epsbased,copyright,final,printable,covers,extendedindex,firstnumbered,tfg,gnuplot]{tfgtfmthesisuam}
\usepackage{needspace}
\usepackage{mdwlist}
\usepackage{amsmath}
\usepackage{physics}
\usepackage{enumitem}
\usepackage{graphicx}  % includegraphics
\setlist[enumerate]{label*=\arabic*.}

\advisor{Francisco Gómez Arribas}

\levelin{Ingeniería Informática}
\title{Estudio de la aplicación de algoritmos cuánticos de optimización con las tecnologías cuánticas actuales}
\author{Oriol Julián Posada}
\privateaddress{C\textbackslash\ Francisco Tomás y Valiente Nº 11}
\copyrightdate{3 de Noviembre de 2017}

\datadir{data}
\graphicspath{{./img}}
\graphicsdir{img}
\logosdir{img}

\resumenfile{inicio/resumen}
\abstractfile{inicio/abstract}

\keywords{Algunas}
\palabrasclave{Otras}

\coverdata
{
  Escuela Politécnica Superior \\
  Universidad Autónoma de Madrid \\
  C\textbackslash Francisco Tomás y Valiente nº 11
}

\begin{document}

\chapter{Introducción\label{cap:introduccion}}{introduccion/introduccion}

\chapter{Estado del arte\label{CAP:ESTADODELARTE}}{estadodelarte/estadodelarte}

\chapter{Diseño\label{CAP:DISEÑO}}{disenyo/disenyo}

\chapter{Desarrollo\label{CAP:DESARROLLO}}{desarrollo/desarrollo}

\chapter{Integración, pruebas y resultados\label{CAP:RESULTADOS}}{resultados/resultados}

\chapter{Conclusiones y trabajo futuro\label{CAP:CONCLUSIONES}}{conclusiones/conclusiones}

%%%%%%%%%%%%%%%%%%%%%%%%%%%%%%%%%%%%%%%%%%%%%%%%%%%%%%%%%%%%%%%%%%%%%%%%%%%%%%%%


%\chapter{Bibliografía\label{CAP:bibliografia}}{bibliografia}
\newpage
\phantomsection

\addcontentsline{toc}{chapter}{Bibliografía}
\bibliographystyle{IEEEtran}
\bibliography{bibliografia}
\end{document}

%%% Local Variables:
%%% mode: latex
%%% TeX-master: t
%%% End:
