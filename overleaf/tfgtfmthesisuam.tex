\documentclass[epsbased,copyright,final,printable,covers,extendedindex,firstnumbered,tfg,lof,lot]{tfgtfmthesisuam}
\usepackage{needspace}
\usepackage{mdwlist}
\usepackage{amsmath}  % matrix/pmatrix y align
\usepackage{physics}
\usepackage{url}  % Urls y paths
\usepackage{enumitem}
\usepackage{graphicx}  % includegraphics
\usepackage{amssymb}  % R (Real numbers)
\usepackage{listings}  % Codigo

\setlist[enumerate]{label*=\arabic*.}

\advisor{Francisco Gómez Arribas}

\levelin{Ingeniería Informática}
\title[Aplicación de algoritmos cuánticos de optimización con las tecnologías cuánticas actuales]{Estudio de la aplicación de algoritmos cuánticos de optimización con las tecnologías cuánticas actuales}

\author{Oriol Julián Posada}
\privateaddress{C\textbackslash\ Francisco Tomás y Valiente Nº 11}
\copyrightdate{3 de Noviembre de 2017}

\datadir{data}
\graphicspath{{./img}}
\graphicsdir{img}
\logosdir{img}

\resumenfile{inicio/resumen}
\abstractfile{inicio/abstract}

\palabrasclave{Computación cuántica, Problemas QUBO, Ising, QAOA, Quantum annealing, optimización cuadrática, Camino más corto, Max cut}
\keywords{Quantum computing, QUBO problems, Ising, QAOA, Quantum annealing, quadratic optimization, Shortest path, Max cut}

\coverdata{
  Escuela Politécnica Superior \\
  Universidad Autónoma de Madrid \\
  C\textbackslash{}Francisco Tomás y Valiente nº 11
}

\begin{document}

\chapter{Introducción\label{cap:introduccion}}{introduccion/introduccion.tex}

\chapter{Estado del arte\label{CAP:ESTADODELARTE}}{estadodelarte/estadodelarte.tex}

\chapter{Diseño\label{CAP:DISEÑO}}{disenyo/disenyo.tex}

\chapter{Desarrollo\label{CAP:DESARROLLO}}{desarrollo/desarrollo.tex}

\chapter{Integración, pruebas y resultados\label{CAP:RESULTADOS}}{resultados/resultados.tex}

\chapter{Conclusiones y trabajo futuro\label{CAP:CONCLUSIONES}}{conclusiones/conclusiones.tex}

%%%%%%%%%%%%%%%%%%%%%%%%%%%%%%%%%%%%%%%%%%%%%%%%%%%%%%%%%%%%%%%%%%%%%%%%%%%%%%%%


%\chapter{Bibliografía\label{CAP:bibliografia}}{bibliografia}
\newpage
\phantomsection{}

\addcontentsline{toc}{chapter}{Bibliografía}
\bibliographystyle{IEEEtran}
\bibliography{bibliografia.bib}

\appendix

\chapter{Conceptos básicos}{apendice/conceptos_basicos.tex}
\chapter{Ejemplo de aplicación de operadores de qaoa\label{CAP:EJEMPLO_QAOA}}{apendice/ejemplo_qaoa.tex}
\chapter{Desarrollo de operaciones}{apendice/operaciones.tex}
\chapter{Paso de función clásica a hamiltoniano\label{CAP:F_CLASICA_A_HAMILTONIANO}}{apendice/f_clasica_a_hamiltoniano.tex}
\chapter{Grafo de camino más corto {--} de función clásica a hamiltoniano del problema\label{CAP:DESARROLLO_ZHIQIANG_ISING}}{apendice/desarrollo_zhiqiang_ising.tex}
\chapter{Modificaciones de qaoa para el camino más corto\label{CAP:OTRAS_SOLUCIONES}}{apendice/resultados_modificaciones_shortest_path.tex}
\chapter{Repositorio de código\label{CAP:github}}{apendice/github.tex}

\end{document}

%%% Local Variables:
%%% mode: latex
%%% TeX-master: t
%%% End:
