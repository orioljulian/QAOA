\section{Tutorial de Qiskit - Max Cut}
\label{sec:4-tutorial de qiskit}

Para asegurar una implementación correcta del algoritmo QAOA primero se aplica al problema presentado en el tutorial de Qiskit.
\cite{qiskit_tutorial_antiguo}
% TODO: Citar el link al cuaderno de jupyter (en github)
Este problema consiste en resolver MAX-CUT para el grafo \ref{fig:4-qiskit grafo}, que consiste en, asignando un valor 0 o 1 a cada nodo maximizar la cantidad de aristas entre nodos de distinto valor.

\begin{figure}{fig:4-qiskit grafo}{}
  \centering
  \includegraphics[scale=0.75]{qiskit-grafo/qiskit-grafo}
\end{figure}

Se parte de este caso por tratarse de uno sencillo. Esto es porque la cantidad de qubits es pequeña y no existen restricciones en este problema, por lo que el espacio de estados válidos disponibles es de $2^n$, siendo \textit{n} el número de qubits del sistema. Estas condiciones hacen que el algoritmo trabaje sobre un circuito pequeño, lo que reduce el ruido, y fácil de implementar.

\begin{itemize}
\item \textbf{Formulación:} \\
  Sea $G = (V, E)$ el grafo de la figura \ref{fig:4-qiskit grafo}, con $V = [0, 1, 2, 3]$ y $E = [(0, 1), (1, 2), (2, 3), (0, 3)]$ los nodos y aristas de G respectivamente

\item \textbf{Objetivo:} \\
  \begin{align} \label{eq:4-tutorial qiskit objetivo}
    max(\sum_{(i, j) \in E} (x_i * (1 - x_j) + x_j * (1 - x_i)))
  \end{align}

\end{itemize}

Como \textit{QAOA} sirve para encontrar el mínimo de una función, se tomará la ecuación [\ref{eq:4-tutorial qiskit objetivo}] $*-1$.

De esta forma se tiene que la función de coste a minimizar para este problema en particular es la siguiente:

\begin{align*}
  f(x) = \sum_{(i, j) \in E} (x_i * (x_j - 1) + x_j * (x_i - 1))
\end{align*}

El cambio de variable ($x_i \rightarrow \frac{1 - z_i}{2}$) de acuerdo con la sección  % TODO: Citar
genera la siguiente función en formato Ising:

\begin{align*}
  g(z) &= \sum_{(i, j) \in E} (\frac{1 - z_i}{2} * (\frac{1 - z_j}{2} - 1) + \frac{1 - z_j}{2} * (\frac{1 - z_i}{2} - 1)) = \\
       &= \sum_{(i, j) \in E} \frac{z_iz_j - 1}{2}
\end{align*}

Para obtener el operador C se tienen que tener en cuenta que debido al postulado de medición en mecánica cuántica \cite{Nielsen_Chuang_2010} la fase global es despreciable. Esto significa que dado un operador lineal $A$ y $n \in {\rm I\!R}$: \\
\(e^{i \gamma n} \cdot e^{i \gamma A} = e^{i \gamma A}\).

También se emplean las siguientes definiciones:  % TODO: Mover las definiciones de puertas al apéndice. Tienen que ser referenciadas por distintas secciones
\begin{itemize}
\item \( Rz_i(\lambda) = exp(-i\frac{\lambda}{2}\sigma_i^z) \)
\item \( Rz_iz_j(\lambda) = exp(-i\frac{\lambda}{2}\sigma_i^z \otimes \sigma_j^z) \)
\end{itemize}

De esta forma:

\begin{align*}
  U(C, \gamma) &=  \exp(-i*\gamma*C) = \exp(-i*\gamma* \sum_{(i, j) \in E} \frac{\sigma_i^z \otimes \sigma_j^z - 1}{2}) \\
          &= \prod_{(i, j) \in E} \exp(-i*\gamma* \frac{\sigma_i^z \otimes \sigma_j^z - 1}{2}) \\
          &= \prod_{(i, j) \in E} [ \exp(-i*\frac{\gamma}{2}* \sigma_i^z \otimes \sigma_j^z) * \exp(i*\frac{\gamma}{2}) ] \\
          &= \prod_{(i, j) \in E} Rz_iz_j(\gamma)
\end{align*}

\paragraph{Circuito obtenido.}
Con el operador \(U(B, \beta)\) y el vector inicial, definidos en la sección \ref{sec:3-circuito de qaoa}, y el operador \(U(C, \gamma)\) obtenido se puede construir el circuito cuántico.
% TODO: insertar imagen del circuito

\section{Grafo simple - Camino más corto}
\label{sec:4-primer grafo}

\begin{figure}[htbp]{}{}
  \centering
  \includegraphics[scale=0.75]{primer-grafo/primer-grafo}
\end{figure}

Con este grafo se pretenden analizar los resultados obtenidos en la sección 2.2 (\textit{Single-Objective Quantum Routing Optimization}) del artículo \cite{multi-objective_routing_optimization}.

El problema a resolver es encontrar el camino más corto que conecte los nodos \textit{0} y \textit{3}.
\begin{itemize}
\item \textbf{Objetivo:}

  \begin{align*}
    &min(5X_{01} + 8X_{02} + 2X_{12} + 7X_{13} + 4X_{23}) \\
    &\textnormal{dde } X_{ij} = \begin{cases}
      1 \textnormal{ si el camino contiene la arista del nodo \textit{i} al \textit{j}} \\
      0 \textnormal{ en otro caso}
    \end{cases}
  \end{align*}
  
\item \textbf{Restricciones:}
  También se deben añadir una serie de restricciones para evitar caminos triviales o incongruentes.

  \begin{align}
    X_{01} + X_{02} = 1
  \end{align}
  \begin{align}
    \begin{split}
      X_{01} = X_{12} + X_{13} \\
      X_{02} + X_{12} = X_{23}
    \end{split}
  \end{align}

  Estas restricciones se pueden justificar como:
  \begin{enumerate}
  \item Debe haber exactamente un eje del camino que involucre al nodo de comienzo. Obliga al camino a comenzar por dicho nodo.
  \item Para cada nodo intermedio debe haber en el resultado tantas aristas entrantes como salientes. Evita caminos incongruentes y hace que el único nodo posible para finalizar sea el nodo final.
    
  \end{enumerate}
  Siguiendo el caso del artículo se elige como valor del \textit{modificador de Lagrange} \(P=27\), ya que debe ser estrictamente mayor que el máximo de la función objetivo:
  \(\max_{x}{f_{\textnormal{sin restricc}}(x)} = \sum_{(i, j)\in{E}}{w_{ij}} = 26\)

\end{itemize}

De acuerdo con los pasos descritos en la sección \ref{sec:3-problemas de optimizacion combinatoria}, la función de coste clásica en su versión QUBO es:

\begin{align*}
  f(X) = &5X_{01} + 8X_{02} + 2X_{12} + 7X_{13} + 4X_{23} + &&\\
         &P(X_{01} + X_{02} - 1)^2 + P(X_{01} - X_{12} - X_{13})^2 + P(X_{02} + X_{12} - X_{23})^2
\end{align*}

El número de qubits del sistema cuántico es igual a la cantidad de variables de la función de coste, esto es, la cantidad de ejes del grafo. \\
Las variables X de la función de coste tienen valores 0 o 1 y para la conversión a su correspondiente función de coste implementada en un circuito cuántico se utiliza la formulación Ising.  % TODO: Citar Ising formulation for many np problems????
Estas nuevas variables ($z$) van a tomar valores -1 y 1 y cada variable $z_k$ estará asociada al qubit k-ésimo del circuito. En este caso la correspondencia entre $X_{ij}$ y $z_k$ es la siguiente:
$X_{01}$ corresponde con $z_0$,
$X_{02}$ corresponde con $z_1$,
$X_{12}$ corresponde con $z_2$,
$X_{13}$ corresponde con $z_3$,
$X_{23}$ corresponde con $z_4$.

Como ya se ha visto en la sección  % TODO: Citar sección correspondiente de Diseño
cada variable $z_i$ va a corresponder con una puerta Pauli-Z en el qubit i.

De acuerdo con la sección \ref{sec:3-operador c} la versión Ising de la función de coste queda como:

\begin{align*}
  g(z) = &5\frac{1-z_0}{2} + 8\frac{1-z_1}{2} + 2\frac{1-z_2}{2} + 7\frac{1-z_3}{2} + 4\frac{1-z_4}{2} + &&\\
         &P(\frac{1-z_0}{2} + \frac{1-z_1}{2} - 1)^2 + P(\frac{1-z_0}{2} - \frac{1-z_2}{2} - \frac{1-z_3}{2})^2 + \\
         &P(\frac{1-z_1}{2} + \frac{1-z_2}{2} - \frac{1-z_4}{2})^2 = \\
       = & 11z_0 - 17.5z_1 - 28z_2 - 17z_3 + 11.5z_4 + \\
         &13.5(z_0z_1 - z_0z_2 - z_0z_3 + z_1z_2 - z_1z_4 + z_2z_3 - z_2z_4) + \\
         &80.5 \\
\end{align*}
\par
Esta igualdad solo se cumple para variables con valores \(\{-1, 1\}\), ya que \(z_i^2 = 1\). \\

\par
De forma similar a la sección \ref{sec:4-tutorial de qiskit} se obtiene el hamiltoniano $U(C, \gamma)$ a partir de $g(z)$:

\begin{align*}
  U&(C, \gamma) = exp(-i \gamma C) = &&\\
   &Rz_0(11*2\gamma) \cdot Rz_1(-17.5*2\gamma) \cdot Rz_2(-28*2\gamma) \cdot Rz_3(-17*2\gamma) \cdot Rz_4(11.5*2\gamma) \cdot \\
   &Rz_0z_1(+13.5 * 2\gamma) \cdot Rz_0z_2(-13.5 * 2\gamma) \cdot Rz_0z_3(-13.5 * 2\gamma) \cdot Rz_1z_2(+13.5 * 2\gamma) \cdot \\
   &Rz_1z_4(-13.5 * 2\gamma) \cdot Rz_2z_3(+13.5 * 2\gamma) \cdot Rz_2z_4(-13.5 * 2\gamma)
\end{align*}

Con el operador \(U(B, \beta)\) y el vector inicial, definidos en la sección \ref{sec:3-circuito de qaoa}, y el operador \(U(C, \gamma)\) obtenido se puede construir el circuito cuántico.

\begin{itemize}
\item \textbf{ Circuito obtenido ($p=1$): }

  \begin{figure}[htbp]{}{}
    \centering
    \includegraphics[scale=0.27]{circuits/primer/primer-circuit-2gamma-p1}
  \end{figure}


\item \textbf{Circuito del artículo ($p=1$):}

  \begin{figure}[htbp]{}{}
    \centering
    \includegraphics[scale=0.3]{circuits/paper/paper-circuit}
  \end{figure}
\end{itemize}

\subsection{Diferencias con el artículo}
El circuito obtenido teóricamente difiere del que se puede ver en la sección 2.2 del artículo \cite{multi-objective_routing_optimization}. \\
En la sección anterior se obtienen operadores con la forma \(Rz(n*2\gamma)\), mientras que en la imagen del circuito en el artículo aparecen como \(Rz(n)\).

Debido a esto, y como se busca replicar los resultados del artículo, se modifica el hamiltoniano para que sea igual al de la imagen.

% TODOO: explicar el porqué del *2 puede ser válido, pero el gamma no
%        C y 2*C tendrían el mismo estado fundamental, pero con distinta energía).
%        \gamma es incorrecto, demostrado con la construcción usando QAOAAnsatz.
%        *2:
%        \begin{align*}
%          \frac{g(z)}{2} = & \frac{11}{2}z_0 - \frac{17.5}{2}z_1 - \frac{28}{2}z_2 - \frac{17}{2}z_3 + \frac{11.5}{2}z_4 + \\
%                           & \frac{13.5}{2}(z_0z_1 - z_0z_2 - z_0z_3 + z_1z_2 - z_1z_4 + z_2z_3 - z_2z_4) + \\
%                           & \frac{80.5}{2} \\
%          \frac{f(X)}{2} = &2.5X_{01} + 4X_{02} + 1X_{12} + 3.5X_{13} + 2X_{23} + &&\\
%                           &\frac{P}{2}(X_{01} + X_{02} - 1)^2 + \frac{P}{2}(X_{01} - X_{12} - X_{13})^2 + \frac{P}{2}(X_{02} + X_{12} - X_{23})^2
%        \end{align*}

%%% Local Variables:
%%% mode: latex
%%% TeX-master: "../tfgtfmthesisuam"
%%% End:
