En las ejecuciones de QAOA, con el fin de medir la eficacia de los resultados obtenidos entre métricas distintas, se han realizado los siguientes tipos de pruebas:

\begin{itemize}
\item \textbf{Estadística máxima:}
  Con este método se busca obviar el ruido presente en cada ejecución. Para ello se realizan \textit{n} iteraciones distintas sobre el algoritmo y para cada una de ellas:
  \begin{enumerate}
  \item
    Se ejecuta el optimizador clásico para hallar los parámetros óptimos (esto supone la ejecución del circuito cuántico el número de veces necesario para que el optimizador encuentre un mínimo local).
  \item
    Se ejecuta el circuito una vez más con los parámetros óptimos.
  \item \label{it:5-definicion estadistica max}
    Se obtiene el camino dado por el algoritmo para recorrer el grafo y se añade dicho camino a un diccionario para su posterior revisión. En el caso de la figura \ref{fig:5-primer-grafo/sin restriccion extra/primer paper aer resultado}
    el resultado sería \textit{10101}, es decir, el camino con mayor valor.
  \end{enumerate}
  
\item \textbf{Estadística global:}
  A diferencia de la estrategia previamente explicada, al realizar el paso \ref{it:5-definicion estadistica max} se toman todos los caminos resultantes de la ejecución del circuito con los parámetros \(\beta_{opt}\) y \(\gamma_{opt}\).
  De esta forma, una ejecución como la dada en la figura \ref{fig:5-primer-grafo/sin restriccion extra/primer paper aer resultado}
  se ve condicionada por todos los resultados, no únicamente por el camino con valor máximo.
  
\item \textbf{Función gamma:}
  Se ha utilizado para comprobar la forma general que tiene la función \textit{execute\_circuit}, a minimizar por el optimizador clásico. Para ello se han realizado circuitos de una sola capa y se ha mantenido el parámetro \(\beta=1\). Se ha decidido así porque dicho parámetro se encarga del ángulo de rotación de los operadores \(\sigma_{x}\), de construcción trivial en comparación con los operadores dependientes de \(\gamma\).
  % TODO: Referenciar punto del apéndice donde se hable sobre la construcción de las puertas Rx y Rz
  Al variar \(\gamma\) y graficar la función resultante se puede intuir la probabilidad de encontrar mínimos locales en lugar del mínimo global. Esto se traduce como la posibilidad de encontrar un resultado subóptimo para el problema, es decir, que el algoritmo falle.
\end{itemize}

A continuación se mostrarán los resultados de ejecución utilizando ambos paradigmas, esto es, QAOA y Quantum Annealing, además de explorar el rendimiento de las ejecuciones en Qiskit variando los métodos para construir la función de coste.

\section{Tutorial de Qiskit - Max Cut}{resultados/tutorial_qiskit}
\section{Grafo simple}{resultados/primer_grafo}
\section{Tercer grafo}{resultados/zhiqiang_grafo}  % TODO: Cambiar nombre

%%% Local Variables:
%%% mode: latex
%%% TeX-master: "../tfgtfmthesisuam"
%%% End:
