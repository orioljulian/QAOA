En las ejecuciones de QAOA, con el fin de medir la eficacia de los resultados obtenidos entre métricas distintas, se han realizado los siguientes tipos de pruebas:

\begin{itemize}
\item \textbf{A/TE}:
  Con este método se busca despreciar el ruido presente en cada ejecución.
  Para ello se realizan \textit{n} ejecuciones distintas del algoritmo para un número \textit{p} de capas y se calcula el porcentaje de ejecuciones en los que se ha obtenido como resultado el camino óptimo del problema.

  \begin{align*}
    \frac{\textnormal{Núm aciertos}}{\textnormal{Total de ejecuciones}}
  \end{align*}

  \paragraph{Ejemplo:}
  Tomando como ejemplo de salida de QAOA la mostrada en la \textit{figura~\ref{fig:5-primer_grafo/sin_restriccion_extra/primer-runtime-mod_paper-1_capa-nairobi_aer}} esto sumaría al cálculo de \textbf{RC/TE} $\frac{1}{n}$, ya que el camino óptimo es el de mayor número de ocurrencias en la ejecución del circuito.
  
\item \textbf{NumOc/TE}:

  Lo que se busca aquí es la media de veces que el camino óptimo aparece en la ejecución del circuito de QAOA\@.

  Con el fin de reducir la aleatoriedad en las operaciones en ordenadores cuánticos, al enviar un circuito para ser ejecutado se realizan varias iteraciones del mismo (cantidad especificada en Qiskit con el \textit{número de ocurrencias}).
  Esto hace que al ejecutar el circuito de QAOA en el último paso con $(\gamma_{opt}, \beta_{opt})$ el resultado mostrado tenga tantos valores como \textit{número de ocurrencias} tuviera la instrucción (por defecto $1024$).

  Ese es el motivo por el que en gráficas como la de la \textit{figura~\ref{fig:5-primer_grafo/sin_restriccion_extra/primer-runtime-mod_paper-1_capa-nairobi_aer}} aparecen varios resultados (aunque en ese caso lo que aparece en el eje vertical es el número de ocurrencias de cada resultado normalizada).

  De esta forma la estadística \textbf{NumOc/TE} se calcula como

  \begin{align*}
    \frac{1}{\textnormal{Total de ejecuciones}} \sum_0^\textnormal{Total de ejecuciones} (\textnormal{Media de núm ocurrencias del óptimo})
  \end{align*}

  \paragraph{Ejemplo:}

  En la \textit{figura~\ref{fig:5-primer_grafo/sin_restriccion_extra/primer-runtime-mod_paper-1_capa-nairobi_aer}} la ``Media de núm ocurrencias del óptimo'' sería 0.59
  
\item \textbf{Función gamma:}
  Se ha utilizado para comprobar la forma general que tiene la función \textit{execute\_circuit}, a minimizar por el optimizador clásico. Para ello se han realizado circuitos de una sola capa y se ha mantenido el parámetro \(\beta=1\). Se ha decidido así porque dicho parámetro se encarga del ángulo de rotación de los operadores \(\sigma_{x}\), de construcción trivial en comparación con los operadores dependientes de \(\gamma\).
  % TODO: Referenciar punto del apéndice donde se hable sobre la construcción de las puertas Rx y Rz
  Al variar \(\gamma\) y graficar la función resultante se puede intuir la probabilidad de encontrar mínimos locales en lugar del mínimo global. Esto se traduce como la posibilidad de encontrar un resultado subóptimo para el problema, es decir, que el algoritmo falle.
\end{itemize}

A continuación se mostrarán los resultados de ejecución utilizando ambos paradigmas, esto es, QAOA y Quantum Annealing, además de explorar el rendimiento de las ejecuciones en Qiskit variando los métodos para construir la función de coste.

\newpage

\section{MAX-CUT en grafo de 4 aristas}{resultados/tutorial_qiskit.tex}

\newpage

\section{Camino más corto en grafo de 4 nodos\label{sec:5-primer_grafo}}{resultados/primer_grafo.tex}

\newpage

\section{Camino más corto para estudiar la variación con el número de capas}{resultados/zhiqiang_grafo.tex}

%%% Local Variables:
%%% mode: latex
%%% TeX-master: "../tfgtfmthesisuam"
%%% End:
