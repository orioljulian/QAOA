En las ejecuciones de QAOA, con el fin de medir la eficacia de los resultados obtenidos entre métricas distintas, se han realizado los siguientes tipos de pruebas:

Debido a la naturaleza probabilística de la computación cuántica, toda ejecución de un circuito se debe realizar varias veces, para obtener una aproximación de la probabilidad de los valores del estado.
Así, cuando se habla del resultado de la ejecución de un circuito cuántico, como el mostrado en la \textit{figura~\ref{fig:5-primer_grafo/sin_restriccion_extra/primer-runtime-mod_paper-1_capa-nairobi_aer}},
se refiere a probabilidades asociadas a cada valor, que han sido calculadas mediante sucesivas ejecuciones del programa.
La cantidad de veces que se ejecuta cada programa, viene dado por las \textit{muestras} (\textit{shots}).

\begin{enumerate}
\item\label{it:5-na} \textbf{NA/TE}:
  Con este método se busca despreciar el ruido presente en cada ejecución.
  Para ello se realizan \textit{n} ejecuciones distintas del algoritmo para un número \textit{p} de capas y se calcula el porcentaje de ejecuciones en los que se ha obtenido como resultado el camino óptimo del problema.

  \begin{align*}
    \textnormal{NA/TE} = \frac{\textit{NA}}{\textit{TE}}
  \end{align*}

  Donde \textit{NA} (número de aciertos) es la cantidad de ejecuciones en las que se ha encontrado el camino óptimo y \textit{TE} es el total de ejecuciones realizadas.

  \paragraph{Ejemplo:}
  Tomando como ejemplo de salida de QAOA la mostrada en la \textit{figura~\ref{fig:5-primer_grafo/sin_restriccion_extra/primer-runtime-mod_paper-1_capa-nairobi_aer}} esto sumaría al cálculo de \textbf{NA/TE} $\frac{1}{n}$, ya que el camino óptimo es el obtenido en la mayor cantidad de muestras en la ejecución del circuito.
  
\item\label{it:5-mm} \textbf{MM/TE}:

  Lo que se busca con esta métrica es la media de veces que el camino óptimo aparece en la ejecución del circuito de QAOA\@.

  Como ya ha sido explicado, al ejecutar el circuito de QAOA en el último paso con $(\gamma_{opt}, \beta_{opt})$ el resultado obtenido tendrá tantos valores como \textit{muestras} tenga la instrucción (por defecto 1024)

  Es el caso de gráficas como la de la \textit{figura~\ref{fig:5-primer_grafo/sin_restriccion_extra/primer-runtime-mod_paper-1_capa-nairobi_aer}}
  en la que el eje vertical muestra la probabilidad aproximada de cada valor.

  De esta forma la estadística \textbf{MM/TE} se calcula como:

  \begin{align*}
    \frac{1}{\textit{TE}} \sum_{i = 1}^\textit{TE} (\textit{MM}_i)
  \end{align*}

  Donde \textit{TE} es el total de ejecuciones y $\textit{MM}_i$ es la media de las muestras en las que se han encontrado el resultado óptimo en la ejecución $i$.

  \paragraph{Ejemplo:}
  En la \textit{figura~\ref{fig:5-primer_grafo/sin_restriccion_extra/primer-runtime-mod_paper-1_capa-nairobi_aer}} la media de muestras del óptimo sería 0.59.
  
\item \textbf{Función gamma:}
  Se ha utilizado para comprobar la forma general que tiene la función \textit{execute\_circuit}, a minimizar por el optimizador clásico.
  Para ello se han realizado circuitos de una sola capa y se ha mantenido el parámetro \(\beta=1\).
  Se ha decidido así porque dicho parámetro se encarga del ángulo de rotación de los operadores \(\sigma_{x}\), de construcción trivial en comparación con los operadores dependientes de \(\gamma\).
  % TODO: Referenciar punto del apéndice donde se hable sobre la construcción de las puertas Rx y Rz
  Al variar \(\gamma\) y graficar la función resultante se puede intuir la probabilidad de encontrar mínimos locales en lugar del mínimo global.
  Esto se traduce como la posibilidad de encontrar un resultado subóptimo para el problema, es decir, que el algoritmo falle.
\end{enumerate}

A continuación se mostrarán los resultados de ejecución utilizando ambos paradigmas, esto es, QAOA y Quantum Annealing, además de explorar el rendimiento de las ejecuciones en Qiskit variando los métodos para construir la función de coste.

Para las ejecuciones de QAOA se utilizarán dos métodos distintos:
una implementación basada en los cálculos desarrollados en la \textit{sección~\ref{CAP:DESARROLLO}} y un enfoque de caja negra, en el que se empleará la función \textit{QAOAAnsatz} de \textit{Qiskit} para generar un circuito cuántico de manera automática.
Este último método se utilizará para comparar con el primero, para así verificar el funcionamiento de la implementación realizada.

\section{MAX-CUT en grafo de 4 aristas}{resultados/tutorial_qiskit.tex}

\section{Camino más corto en grafo de 4 nodos\label{sec:5-primer_grafo}}{resultados/primer_grafo.tex}

\section{Camino más corto para estudiar la variación con el número de capas}{resultados/zhiqiang_grafo.tex}

%%% Local Variables:
%%% mode: latex
%%% TeX-master: "../tfgtfmthesisuam"
%%% End:
