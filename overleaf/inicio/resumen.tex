La computación cuántica, que aplica principios de la mecánica cuántica a la computación, promete ser una herramienta para tratar problemas que, por la gran cantidad de datos o por la dimensión del espacio de búsqueda, no son resolubles actualmente con los recursos disponibles en computación clásica, a la vez que se consigue disminuir el tiempo de ejecución.

Este TFG se desarrolla en el contexto de los problemas de optimización y estudia la aplicación de algoritmos en los que se busca el extremo de una función de coste.
Se ha evaluado la utilización de dos tecnologías cuánticas diferentes para tratar de mejorar las soluciones clásicas. 

Concretamente, se han utilizado tres problemas de optimización para validar el funcionamiento híbrido del algoritmo QAOA (\textit{Quantum Approximate Optimization Algorithm}) que combina computación cuántica, para obtener el resultado óptimo de la función de coste, y computación clásica, para encontrar los parámetros que optimizan el funcionamiento del circuito que calcula dicha función de coste.

Se han comparado  los resultados obtenidos con QAOA en ordenadores cuánticos generalistas, con los obtenidos por QA (\textit{Quantum Annealing}) en sistemas cuánticos D-Wave, específicos para problemas de optimización.

Adicionalmente se han realizado análisis estadísticos de los resultados obtenidos, comparando los valores publicados en dos artículos internacionales con los que se obtienen en este TFG y con los que resultan de utilizar métodos de librería disponibles en el entorno de programación Qiskit.
Para esto, se ha realizado la implementación completa del algoritmo QAOA\@.

Este TFG ha sido desarrollado con dos objetivos distintos:

\begin{itemize}
\item Un objetivo formativo y didáctico, con una explicación completa de cómo y por qué funciona el algoritmo QAOA y cómo se compara con QA\@.

\item La evaluación de la aplicación práctica de estos algoritmos con las tecnologías cuánticas actuales, realizando un análisis estadístico de la validez de los resultados obtenidos.
\end{itemize}


%%% Local Variables:
%%% mode: latex
%%% TeX-master: "../tfgtfmthesisuam"
%%% End:
