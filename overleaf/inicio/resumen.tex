En el contexto actual, donde la presencia de tecnologías basadas en \textit{Big Data} es cada vez mayor, es muy común que problemas que antes se consideraban triviales sean ahora irresolubles en la práctica por su magnitud.
\\
Es crucial encontrar los métodos más eficientes para resolver estos problemas, ya que no solo hacen que un problema pueda ser resuelto con un coste menor, sino que problemas irresolubles puedan ser resueltos.
\\\\
El paradigma de la computación cuántica, que aplica principios de la mecánica cuántica a la computación, promete ser una herramienta para tratar estos problemas con un coste temporal menor al actual.
\\
Concretamente en el contexto de los problemas de optimización, en los que se busca el extremo de una función de coste, existen numerosos algoritmos que utilizan tecnologías cuánticas para tratar de mejorar las soluciones clásicas.
\\
Es por esto que en este trabajo se realiza un estudio en profundidad sobre el funcionamiento y rendimiento de uno de estos algoritmos, QAOA (\textit{Quantum Approximate Optimization Algorithm})\@.
\\
Para ello se ha realizado una implementación completa del algoritmo y se ha aplicado a problemas de optimización obtenidos de diferentes fuentes.
\\
Además, se ha utilizado el algoritmo de QA (\textit{Quantum Annealing}) para comparar con los resultados dados por QAOA a estos problemas.
\\\\
De esta forma se ha obtenido un trabajo con dos fines distintos.
\begin{itemize}
\item Un fin didáctico, por el cual se ha realizado una explicación completa de cómo y por qué funciona QAOA\@, dando también una construcción sistematizada del algoritmo.

\item Un fin estadístico, por el que se ha explorado cómo se comporta el algoritmo en distintas situaciones.
\end{itemize}


%%% Local Variables:
%%% mode: latex
%%% TeX-master: "../tfgtfmthesisuam"
%%% End:
