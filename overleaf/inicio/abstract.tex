In the current context, in which the increase of technologies based in \textit{Big Data} is getting bigger, it is not unusal having problems considered trivial now being irresolvable in practice, due to their magnitude.
\\
It is crucial to find the most efficient methods for solving this problems, as they do not only make it possible to solve a problem with a lower cost, but to solve now unsolvable problems.
\\\\
The paradigm of quantum computation, which applies principles of quantum mechanics to computation, promises to be a tool to manage these problems with a time cost lower than the actual.
\\
Specifically in the context of optimization problems, in which the maximum or minimum of a cost function is searched, there are several algorithms that use quantum technologies to try to improve the already existing classical solutions.
\\
This is why in this writing an in-depth study about one of this alorithms, QAOA (\textit{Quantum Approximate Optimization Algorithm}) has been conducted.
\\
To acomplish this, a complete implementation of the algorithm has been developed and tried against optimization problems obtained from different sources.
\\
Also, the QA (\textit{Quantum Annealing}) algorithm has been used to compare with the results of this problems obtained by QAOA\@.
\\\\
Thus, the work developed has two main purposes.
\begin{itemize}
\item A didactic purpose, for which a complete explanation of how and why QAOA works has been made, giving also a systematized construction of the algorithm.
  
\item A statistical purpose, for which the behaviour of the algorithm in different situations has been explored.
\end{itemize}


%%% Local Variables:
%%% mode: latex
%%% TeX-master: "../tfgtfmthesisuam"
%%% End:
