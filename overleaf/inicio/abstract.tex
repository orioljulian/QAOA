The paradigm of quantum computation, which applies principles of quantum mechanics to computation, promises to be a tool to manage problems that, due to the big amount of data or the extent of the search space, are not solvable at the moment with the available resources in classic computation, at the same time that a reduction of the execution time is achieved.

This end-of-degree project is developed in the context of optimization problems and studies the application of algorithms in which the maximum or minimum of a cost function is searched.
Trying to improve the available classical solutions, two quantum technologies have been used.

Specifically, three optimization problems have been used to validate the hybrid functioning of QAOA (\textit{Quantum Approximate Optimization Algorithm}), that combines quantum computing, to obtain the optimal result of the cost function, and classical computing, to find the parameters that optimize the functioning of the circuit that evaluates said cost function.

The result obtained with QAOA in generic quantum computers has been compared with the results obtained by QA (\textit{Quantum Annealing}) in D-Wave's quantum systems, specifically designed to solve optimization problems.

Additionally, statistical analysis have been carried out on the obtained results, comparing the values published in two international articles with the ones obtained in this work and with the ones resulting from using methods available in the Qiskit programming environment.

This end-of-degree project has been developed with two different objectives:

\begin{itemize}
\item A training and didactic objective, with a complete explanation of how and why QAOA works and how it compares with QA\@.

\item The evaluation of the practical application of these algorithms with current quantum technologies, doing a statistical analysis to estimate the validity of the obtained results.
\end{itemize}


%%% Local Variables:
%%% mode: latex
%%% TeX-master: "../tfgtfmthesisuam"
%%% End:
