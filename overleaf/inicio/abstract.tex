Nowadays, as the volume of data on the internet increases exponentially, cyber attacks such as distributed denial-of-service, are becoming increasingly common.
\\\\The development of modular security tools that, on the one hand, allow easy addition of contributions from the developer community and, on the other hand, are user-friendly is currently of interest.
\\\\The contribution of this end-of-degree project consists in the design and development of a firewall, which will allow, on the one hand, programmers to add their own solutions in a very simple way and, on the other hand, users to manage the firewall with ease.
\\\\First, the existing firewall solutions and systems on the market will be analysed in order to identify their strengths and weaknesses. The purpose is to design an application that will address these shortcomings and be truly useful to the user.
\\\\Once the study is completed, the application will be designed using the best system for filtering packages. Among the various possibilities currently available, XDP is a very interesting alternative since it allows the code to be executed using the network card driver. The advantage of this system is its ability to eliminate packages at high speed and with less resources, which is very important when it comes to stopping an attack. The main disadvantage is the complexity of the development and communication due to the need to use low level languages in the internal functioning and high level languages in the interface.
\\\\Finally, performance and validation tests will be carried out in order to verify that all requirements are met and the firewall is viable. This is very important, since it is an application running in a critical area of the system.

\keywords{Firewall, agile methodologies, packet filtering, XDP, eBPF, Flask Python, cyber attacks, API REST, web development}
